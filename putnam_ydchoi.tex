\documentclass{article} % For LaTeX2e
\usepackage{nips14submit_e,times}
\usepackage{amsmath}
\usepackage{amsthm}
\usepackage{amssymb}
\usepackage{mathtools}
\usepackage{hyperref}
\usepackage{url}
\usepackage{algorithm}
\usepackage[noend]{algpseudocode}
%\documentstyle[nips14submit_09,times,art10]{article} % For LaTeX 2.09

\usepackage{graphicx}
\usepackage{caption}
\usepackage{subcaption}

\def\eQb#1\eQe{\begin{eqnarray*}#1\end{eqnarray*}}
\def\eQnb#1\eQne{\begin{eqnarray}#1\end{eqnarray}}
\providecommand{\e}[1]{\ensuremath{\times 10^{#1}}}
\providecommand{\pb}[0]{\pagebreak}



\newenvironment{claim}[1]{\par\noindent\underline{Claim:}\space#1}{}
\newtheoremstyle{quest}{\topsep}{\topsep}{}{}{\bfseries}{}{ }{\thmname{#1}\thmnote{ #3}.}
\theoremstyle{quest}
\newtheorem*{definition}{Definition}
\newtheorem*{theorem}{Theorem}
\newtheorem*{question}{Question}
\newtheorem*{exercise}{Exercise}
\newtheorem*{challengeproblem}{Challenge Problem}
\newtheorem*{solution}{Solution}
\newtheorem*{remark}{Remark}
\usepackage{verbatimbox}
\usepackage{listings}
\title{Putnam Compendium}


\author{
Youngduck Choi \\
CILVR Lab \\
CIMS, New York University \\
\texttt{yc1104@nyu.edu} \\
}


% The \author macro works with any number of authors. There are two commands
% used to separate the names and addresses of multiple authors: \And and \AND.
%
% Using \And between authors leaves it to \LaTeX{} to determine where to break
% the lines. Using \AND forces a linebreak at that point. So, if \LaTeX{}
% puts 3 of 4 authors names on the first line, and the last on the second
% line, try using \AND instead of \And before the third author name.

\newcommand{\fix}{\marginpar{FIX}}
\newcommand{\new}{\marginpar{NEW}}

\nipsfinalcopy % Uncomment for camera-ready version

\begin{document}


\maketitle

\begin{abstract}
The work contains the solutions to Putnam problems. 
\end{abstract}

\section{Solutions}

\begin{question}[2012 A-1]
\end{question}
\begin{solution} 
Assume without loss of generality that $d_1 \leq d_2 \leq ... \leq d_n$.
Suppose for sake of contradiction that there does not exist distinct 
indices $i, j, k$ such that $d_i, d_j, d_k$ are the side lengths of an acute
triangle. By the property of an acute traingle we have (PATCH)
\eQnb
d_{i+2}^2 \geq d_{i+1}^2 + d_{i}^2 
\eQne
for all $i$ such that $1 \leq i \leq 10$.
We first claim that
\[
d_i^2 \geq F_i d_1^2 \\
\]
holds for all $1 \leq i \leq 12$ where $F_i$ is a $i$th Fibonacci number with $F_1 = 1$ and $F_2 = 1$. We proceed
to prove the claim by strong induction. Base case of the induction trivially holds as $F_1 = 1$ yielding
\[
d_1^2 \geq d_1^2. 
\]
Now assume that the statement holds true from $1$ to $i$. From $(1)$ we obtain
\[
d_{i+1}^2 \geq d_{i}^2 + d_{i-1}^2.
\]
With the inductive hypothesis we can lower bound the RHS as
\eQnb
d_{i}^2 + d_{i-1}^2 \geq F_i d_{1}^2 + F_{i-1}d_1^2.
\eQne
Factoring and substituting the Fibonacci recurrence to RHS of $(2)$ we obtain
\[
d_{i}^2 + d_{i-1}^2 \geq F_{i+1}d_{1}^2.
\]
Hence, we finally get that
\[
d_{i+1}^2 \geq F_{i+1}d_{i}^2
\]
which completes the induction. Hence the claim $d_i^2 \leq d_1^2$ holds true for all $i$ such that
$1 \leq i \leq 12$.
Notice that for $i=12$ case we have $F_12 = 144$ yielding 
\[
d_{i+1}^2 \geq 144d_1^2.
\]
As the numbers are chosen from the open interval $(1,12)$ we have that $d_{i+1}^2$ is strictly less
than $144$, but $144d_1^2$ is strictly greater than $144$, which is a contradiction. Therefore,
we have shown given any $d_1, d_2, ... , d_{12}$ chosen from $(1,11)$, there exists three distinct 
indices $i, j, k$ such that $d_i, d_j, d_k$ form side lenghts of an acute traingle. $\qed$ 
\end{solution}


\begin{remark}
\end{remark}
\pb
\begin{question}[2012 A-2]
\end{question}
\begin{solution}
Assume that for every $x$ and $y$ in $S$ there exists $z$ in $S$ such that $x * z = y$.
Assume that $a,b,c$ are in $S$ and $a * c = b * c$ holds. Let $d$ be the element such that
$d = a * c = b * c $ holds. 
From the assumption we can deduce that there exist $e$ and $f$ in $S$ such that 
\eQb
d * e &=& a \\
d * f &=& b \\
\eQe

\end{solution}
\begin{question}[2008 A-1]
\end{question}
\begin{solution}
We claim that $g(x) = f(x,0)$ satisfies the given properties.
Substituting $(x,y,z) = (0,0,0)$ into the functional equations yields
\[
f(0,0) + f(0,0) + f(0,0) = 3f(0,0) = 0
\]
which gives that $f(0,0) = 0$. Substituting $(x,y,z) = (0,0,0)$ we obtain
\[
f(x,0) + f(0,0) + f(0,x) = 0;
\]
hence, $f(x,0) = -f(0,x)$. Substituing $(x,y,z) = (x,y,0)$ gives
\[
f(x,y) + f(y,0) + f(0,x) = 0.
\]
Rearranging and substituting $f(x,0) = -f(0,x)$ results in
\[
f(x,y) = f(x,0) - f(y,0) \\
\]
Let $g(x) = f(x,0)$ which is a well-defined function $g: \mathbb{R} \to \mathbb{R}$,
we see that
\[
f(x,y) = g(x) - g(y)
\]
as desired. Hence, we have shown that there exists a function $g: \mathbb{R} \to \mathbb{R}$,
$g(x) = f(x,0)$ in particular, that $f(x,y) = g(x) - g(y)$ is satisfied.
\end{solution}

\bigskip

\begin{question}[2010 A-1]
\end{question}
\begin{solution} 
We claim that $k = \lceil \dfrac{n}{2} \rceil$. 
For $n$ odd, the lower bound can be acheived by the partition
\[
\{n \}, \{1, n-1 \} , \{2 , n-2\}, ...
\]
Since $n-1$ is even, the partition is well defined. Now for the case where
\[
\{n, 1\}, \{n-1, 2 \} , \{n -2 , 3\}, ...
\]
Now we show that such partition strategy is indeed optimal for all $n$.
Let us denote the sum of each box of the optmal strategy as $S$. Since the number $n$ itself
belongs to one of the boxes we obtain a lower bound, $S \geq n$. As the partition strategy
for $n$ odd case yields $S = n$ and $n$ is indeed a lower bound on $S$, we have shown that
the strategy is optiaml for $n$ odd case. 
For $n$ even we show that $S$ has to be lower bounded by $n+1$. Suppose that $S = n$.
A paritition that can acheived $S$ must have the following configuration

\end{solution}

\begin{question}[2010 B-1]
\end{question}
\begin{solution}
Suppose that such sequence exists.
Suppose that $0 \leq a_i^2 \leq 1$ for $i$. Then we get that
\[
\sum_{i=1}^{\infty} a_i^4 \leq \sum_{i=1}^{\infty} a_i^2. 
\]
Since $\sum_{i=1}^{\infty} a_i^k = k$, by substitution we get $4 \leq 2$,
which is a contradiction. Hence, there exists an index $k$ such that $a_k^2 > 1$.
Now, for large enough $m$, we have $a_k^{2m} > 2m$, which is a contradicts the condition
of the sequence.
Therefore, there does not exist an infinite sequence of real numbers
such that 
\[
a_1^{m} + a_2^{m} ... = m
\]
for every positive integer $m$.
\end{solution}

\begin{question}[2001 A-1]
\end{question}
\begin{solution}
Consider a set $S$ and a binary operation $*$ and $S$ is closed under $*$.
Assume that $(a*b)*a = b$ for all $a, b \in S$. Substituting $(b*a)$ into the assumption,
we obtain
\[
((b*a)*b)*(b*a) = b.
\]
From $(b*a)*b = a$, we can re-write the above equality as
\[
a*(b*a) = b,
\]
as desired.
\end{solution}

\begin{question}[2001 A-2]
\end{question}
\begin{solution}
Let us denote the probability of having odd number of heads after $i$ tosses as $P_i$. Then, 
for $i \geq 2$ we can express $P_{i}$ as 
\eQb
P_{i} &=& (1 - C_{i})P_{i-1} + C_{i}(1 - P_{i-1}),
\eQe
where $C_{i}$ is a probability of flipping a head for the $i$th coin. Simplifying
the above equality yields
\eQb
P_{i} &=& (1-2C_{i})P_{i-1} + C_i.
\eQe
Substituting
$C_{i} = \dfrac{1}{2i+1}$ into the last equality, we obtain
\eQb
P_i &=& P_{i-1} + \dfrac{1}{2i+1} - 2
\eQe

\end{solution}

\end{document}

